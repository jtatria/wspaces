\nonstopmode{}
\documentclass[letterpaper]{book}
\usepackage[times,inconsolata,hyper]{Rd}
\usepackage{makeidx}
\usepackage[utf8]{inputenc} % @SET ENCODING@
% \usepackage{graphicx} % @USE GRAPHICX@
\makeindex{}
\begin{document}
\chapter*{}
\begin{center}
{\textbf{\huge Package `wspaces'}}
\par\bigskip{\large \today}
\end{center}
\begin{description}
\raggedright{}
\inputencoding{utf8}
\item[Type]\AsIs{Package}
\item[Title]\AsIs{What the Package Does (Title Case)}
\item[Version]\AsIs{0.1.0}
\item[Author]\AsIs{Who wrote it}
\item[Maintainer]\AsIs{The package maintainer }\email{yourself@somewhere.net}\AsIs{}
\item[Description]\AsIs{More about what it does (maybe more than one line)
Use four spaces when indenting paragraphs within the Description.}
\item[License]\AsIs{What license is it under?}
\item[Encoding]\AsIs{UTF-8}
\item[LazyData]\AsIs{true}
\item[Imports]\AsIs{Rcpp (>= 0.12.11), RcppEigen, Matrix, magrittr}
\item[LinkingTo]\AsIs{Rcpp, RcppEigen}
\item[RoxygenNote]\AsIs{6.0.1}
\item[SystemRequirements]\AsIs{C++11}
\end{description}
\Rdcontents{\R{} topics documented:}
\inputencoding{utf8}
\HeaderA{counts\_to\_factor}{Build a factor from max/min values in a series of variables}{counts.Rul.to.Rul.factor}
%
\begin{Description}\relax
Build a factor from max/min values in a series of variables
\end{Description}
%
\begin{Usage}
\begin{verbatim}
counts_to_factor(d, var, min = FALSE, conf = FALSE)
\end{verbatim}
\end{Usage}
%
\begin{Arguments}
\begin{ldescription}
\item[\code{d}] A data frame

\item[\code{var}] An optional list of variables to compare

\item[\code{min}] If true, use minimum instead of maximum
\end{ldescription}
\end{Arguments}
%
\begin{Value}
A factor of the same length as d, where the value for each observation will correspond
to the name of the column in var that
contains the minimum or maximum value for that row.
\end{Value}
%
\begin{Examples}
\begin{ExampleCode}
d <- as.data.frame( matrix( rnorm( 300 ), nrow = 100, ncol = 3 ) )
names( d ) <- c( 'red', 'blue', 'green' )
count_to_factor( d )

\end{ExampleCode}
\end{Examples}
\printindex{}
\end{document}
